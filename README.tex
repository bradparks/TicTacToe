\documentclass[12pt, fullpage]{article}
\begin{document}

\title{TicTacToe: A Unity Networking Example}
\author{Jake Cohen}
\date{February 11, 2014}
\maketitle

\paragraph{This is the README for a very basic TicTacToe game made in Unity3D, for better understanding and implementation of Unity's networking features with respect to a turn-based (asynchronous) board game. Since TicTacToe is turn-based, only RPC calls are used.\\The game provides online peer-to-peer hosting and connections (play with your friends), as well as a 'local' hotseat mode. In the project, NetworkViews are attached to GameObjects in order to instantiate objects across the network, as well as update data across all players' client scripts through RPC calls. Determining whose clicks are being listened to is done through the use of private data on each end. }

\emph{\\Note: This game is not perfect, but a learning experience. I'd like to share what I've learned with this concrete example. Feel free to contribute to the project where you see fit (that even includes this README).\\}

\paragraph{\underline{URL: madebyjake.co/TTT.html}}

\paragraph{\\Basic Unity networking learned from:}
\begin{description}
\item[Text] http://www.paladinstudios.com/2013/07/10/how-to-create-an-online-multiplayer-game-with-unity/
\item[Video] http://cgcookie.com/unity/2011/12/20/introduction-to-networking-in-unity/
\end{description}

\end{document}




